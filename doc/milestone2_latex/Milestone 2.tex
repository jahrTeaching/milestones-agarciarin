\documentclass[12pt,a4paper]{article}
\usepackage[a4paper, left=2.5cm,right=2.5cm,top=3cm,bottom=3cm]{geometry}
\usepackage[utf8]{inputenc}
\usepackage[spanish]{babel}
\addto\captionsspanish{
	\renewcommand\chaptername{}
	\renewcommand\appendixname{Anexo}
	\renewcommand\appendixpagename{Anexos}
	\def\tablename{Tabla}
	\def\listtablename{\'Indice de tablas}
}
\usepackage{ucs}
\usepackage{subfig}
\usepackage{float}
\usepackage{amsmath}
\usepackage{amsfonts}
\usepackage{amssymb}
\usepackage{graphicx}
\usepackage{listings}
\usepackage{color}
\usepackage[T1]{fontenc}
\usepackage[scaled]{beramono}
\usepackage{upquote}
\usepackage{xcolor}
\usepackage{listings}
\usepackage{caption}
\usepackage{chngcntr}
\usepackage{endnotes}
\usepackage{float}
\usepackage{hyperref}
\usepackage{wrapfig}
\usepackage{fancyhdr}
\usepackage{emptypage}
\usepackage{times}
\usepackage{array}
\usepackage{setspace}
\usepackage{multirow}
%\usepackage{subfigure}
\usepackage{verbatim}
%\usepackage{tabulary}
%\usepackage{graphicx,subfigure}
\usepackage[bottom]{footmisc} 
\usepackage{appendix}
\usepackage{mathtools}
\usepackage{textcomp}


%listing, code style
% Define a custom color
\definecolor{backcolour}{rgb}{0.95,0.95,0.92}
\definecolor{codegreen}{rgb}{0,0.6,0}

% Define listing style
\lstset{
	language=python,
	tabsize=4,
	rulecolor=,
	basicstyle=\scriptsize,
	upquote=true,
	aboveskip={1.2\baselineskip},
	columns=fixed,
	numbers=left,
	showstringspaces=false,
	extendedchars=true,
	breaklines=true,
	prebreak = \raisebox{0ex}[0ex][0ex]{\ensuremath{\hookleftarrow}},
	showtabs=false,
	showspaces=false,
	showstringspaces=false,
	identifierstyle=\ttfamily,
	keywordstyle=\color[rgb]{0,0,1},
	commentstyle=\color[rgb]{0.133,0.545,0.133},
	stringstyle=\color[rgb]{0.627,0.126,0.941},
} 


\renewcommand{\notesname}{Fuentes}
\renewcommand{\baselinestretch}{1.2}
\hypersetup{
	colorlinks,%
	citecolor=black,%
	filecolor=black,%
	linkcolor=black,%
	urlcolor=black
}

\begin{document}
	
\begin{titlepage}
	\centering
	%\null\vfill
	\includegraphics[width=\textwidth]{FIGURES/Portada/Logo_portada.png} 
	\vspace{1.5cm}
	
	Universidad Politécnica de Madrid
	\\Escuela Técnica Superior de Ingeniería Aeronáutica y del Espacio
	\vspace{2cm}
	
	{\large MÁSTER UNIVERSITARIO EN SISTEMAS ESPACIALES}
	\vspace{2cm}
	
	{\LARGE MILESTONE 2}
	\vspace{1cm}
	
	{\large Ampliación de Matemáticas I}
	\vspace{4cm}
	
	\begin{center}
		\large{\textbf{\today}} \\
	\end{center}
	
	Autor: Alberto García Rincón
	\vfill
\end{titlepage}

%****INDICES****
\newpage
\pagestyle{empty}
\tableofcontents	

%******DOC******	
\newpage
\pagenumbering{arabic}
\setcounter{page}{1}
\pagestyle{fancy} 

\section{Introducción}
Como continuación del ejercicio resuelto durante la semana anterior, se ha reestructurado el código referente al programa para resolver el problema de Cauchy aplicado para la resolución de una órbita Kepleriana mediante distintos métodos. Aparte de resolver el problema con el método explícito de Euler y con el método de Runge-Kutta de cuarto orden, que ya se programaron para el anterior ejercicio, se han añadido los métodos de Euler implícito y el método de Crank-Nicolson.

Se procederá a una explicación del código usado para la resolución de este ejercicio y a un breve análisis de los resultados obtenidos.

\section{Código de Python}
Para este ejercicio se ha procedido a una separación del programa principal o $"main"$ (denominado \textsl{Milestone\_2.py}) donde se llaman a los distintos procesos o funciones que se encuentran en otros archivos individuales.

El programa principal contiene las condiciones iniciales del problema a resolver y va llamando a los distintos subprogramas o funciones para obtener los distintos resultados según el método que se use. Por tanto en el programa $"main"$ se tienen todos los datos calculados de los distintos métodos que aquí se implementan. Estos datos se almacenan en una variable, la cual se pasa como argumento a las funciones encargadas de dibujar las gráficas que se encuentran definidas en otro archivo distinto.


\section{Resultados de los métodos}
\subsection{Euler explícito}



\subsection{Runge-Kutta de cuarto orden}



\subsection{Euler implícito o inverso}



\subsection{Crank-Nicolson}



\section{Comparación métodos}



\end{document}
